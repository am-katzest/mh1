\documentclass{article}
\usepackage[a4paper,margin=0.8in]{geometry}

\usepackage{amsmath} \usepackage{amssymb} \usepackage{amsfonts}
\usepackage{pifont}
\usepackage{enumitem}
\usepackage[T1]{fontenc}
\usepackage{lmodern}
\usepackage{rotating}
\usepackage{siunitx}
\usepackage[polish]{babel}
\usepackage[utf8]{inputenc}
\usepackage{multirow}
\usepackage[yyyymmdd]{datetime} \renewcommand{\dateseparator}{-} % ISO-8601
\usepackage{tabularx}
\usepackage{graphicx}
\usepackage{float}
\usepackage{hyperref}
\usepackage{minted}
\newcommand{\cljt}[1]{\mintinline{clojure}{#1}}
\newcommand{\inputgraph}[1]{\newpage \input{#1}\newpage} % p, n w przeglądarce ^w^
\begin{document}
\begin{minipage}{0.35\linewidth}
	\begin{tabular}{lr}
		Antoni Jończyk & 236551 \\
		Tomasz Roske   & 236639
	\end{tabular} \hfill
\end{minipage}
\hfill
\begin{minipage}{0.35\linewidth}
	\hfill Rok akademicki 2022/23 \par
	\hfill czwartek, 13:00
\end{minipage}
\bigskip \bigskip \bigskip \bigskip \bigskip
\begin{center}
	\textbf{Metaheurystyki i ich zastosowania, zadanie 2}\\
	\bigskip
	\large implementacja algorytmu genetycznego
\end{center}
\bigskip \bigskip
\section{Działanie programu}

\subsection{struktury danych}
\begin{itemize}
	\item osobnik \cljt{specimen} zawiera listę cech, wagę, wartość i flagę
	      wyznaczającą poprawność
	\item stan świata \cljt{state} to zbiór osobników
\end{itemize}
\subsection{symulacja}
wynikiem symulacji jest lista stanów świata z poszczególnych pokoleń
\inputminted{clojure}{snippets/alg.clj_simulate}
\inputminted{clojure}{snippets/alg.clj_advance}
początkowe pokolenie zawiera osobniki z cechami z równą szansą na bycie 0 i 1
\inputminted{clojure}{snippets/alg.clj_orphan}
%\inputminted{clojure}{snippets/alg.clj_specimen} % zbyt niski poziom

\subsection{wybór}
do zadecydowania które osobniki przetrwają lub rozmnorzą się wykorzystywana jest
jedna z trzech funkcji; \cljt{ranked} przypisująca osobnikom szansę równą ich
miejscu w rankingu, \cljt{roulette}, przypisująca osobnikom szansę równą ich
przystosowaniu  albo \cljt{top} która zawsze wybiera \cljt{n} najbardziej
przystosowanych osobników
\inputminted{clojure}{snippets/alg.clj_top}
\subsection{funkcja przystosowania}
Postanowiliśmy dać szansę osobnikom, które pomimo nieznacznego przekroczenia dopuszczalnej
wagi mają duży stosunek wartości do wagi.
\href{https://www.desmos.com/calculator/gikud1m6v8}{link do desmosu}
\inputminted{clojure}{snippets/alg.clj_scoring}
\subsection{krzyżowanie}
Funkcje krzyżujące zwracają listę funkcji dwuargumentowych, które są aplikowane
do kolejnych wyborów krzyżowanych osobników, lista wyników tych wywołań tworzy
genom nowego osobnika.
\inputminted{clojure}{snippets/alg.clj_krzyżowanie}



\section{analiza wyników}
\subsection{co to jest i jak to czytać}
Osobniki które przekroczyły maksymalną masę nie są uwzględniane w wynikach.
\\
Na grafach pokazany jest rozkład (pięć kwantyli) maksimów i median każdej
symulacji, oś X reprezentuje pokolenie, a Y wartość.
\\
Nie popierając tego niczym, poza kilkoma tysiącami prób które nie znalazły
lepszego, że globalne maksimum wynosi 13692887.
\\
W tabelach przedstawiono rozkład przystosowania najlepszej utworzonej do danego pokolenia
(nawet jeżeli umarła) jednostki dla poszczególnych symulacji.
\\
W kolumnach min, Q2 i max są odpowiednie wartości (jako procent maksimum
globalnego) danych kwantyli.
\\
W kolumnie \%global jest procent symulacji które znalazły maksimum globalne, a
w \%$\le$1\% jest procent symulacji które znalazły kombinację cech, z wartością w
granicach 1\% od niej.

\subsection{łączenie}
ten parametr decyduje, czy symulacja pozwala na jednoczesne istnienie wielu
identycznych jednostek.\\
w symulacjach \ref{tab:1-mit} każda jednostka musi być wyjątkowa, a w
\ref{tab:1-mif} nie. połączenie się rozkładu mediany i maksimum na wykresie
\ref{tab:1-mif} wskazuje na dominację populacji przez identyczne jednostki
z lokalnego maksimum. Ponieważ w \ref{tab:1-mit} wyniki są lepsze, ten wybór
jest zastosowany przy innych próbach.

\inputgraph{1-mit.transient.tex}
\inputgraph{1-mif.transient.tex}
\subsection{prawdopodobieństwo mutacji}
ten parametr decyduje o względnych szansach mutacji i krzyżowania
\\
W symulacjach \ref{tab:2-m0} do \ref{tab:2-m100}, ze względu na dość małą przeszukiwaną
przestrzeń i ograniczenie zaniku różnorodności genetycznej, nawet przy braku
szansy na mutację populacja zachowuje odpowiednie geny, więc zwiększenie szansy
na mutację ma dość negatywny wpływ.
\inputgraph{2-m0.transient.tex}
\inputgraph{2-m10.transient.tex}
\inputgraph{2-m25.transient.tex}
\inputgraph{2-m50.transient.tex}
\inputgraph{2-m75.transient.tex}
\inputgraph{2-m100.transient.tex}

Symulacje \ref{tab:2-x0} do \ref{tab:2-x100} nie są zabezpieczone przed utratą
różnorodności genetycznej i wykorzystany jest w nich najgorszy pod tym względem
algorytm wybierania. W związku z tym cała populacja bardzo szybko ląduje w
lokalnym maksimum, i przy braku mutacji (\ref{tab:2-x0}) nie wychodzi z niego. w
pozostałych symulacjach ze względu na brak zmian w przypadku krzyżowania
identycznych osobników, tempo ewolucji jest proporcjonalne do szansy na mutację.

\inputgraph{2-x0.transient.tex}
\inputgraph{2-x25.transient.tex}
\inputgraph{2-x50.transient.tex}
\inputgraph{2-x75.transient.tex}
\inputgraph{2-x100.transient.tex}

Sprawdziliśmy też wpływ ilości zmian podczas mutacji na tempo ewolucji, symulacja
\ref{2-x100} (i wszystkie inne poza tą sekcją) ma jednakową szansa na zmianę
dwóch, trzech lub czterech
wyborów osobnika. W symulacjach \ref{tab:2-1100} do \ref{tab:2-4100} mutacja
zmienia zawsze odpowiednio 1, 2, 3 lub 4 wybory, reszta parametrów jest
identyczna z \ref{tab:2-x100}. jak widać, zmienianie jedynie jednej lub dwóch
cech daje kiepskie wyniki (mała szansa na ulepszenie dość dobrego osobnika).
\\

\inputgraph{2-1100.transient.tex}
\inputgraph{2-2100.transient.tex}
\inputgraph{2-3100.transient.tex}
\inputgraph{2-4100.transient.tex}

\subsection{wielkość populacji}
najlepsze wyniki osiągnęły symulacje w których stosunek populacji do tempa wymiany wynosił od 4 do 6,
niższe populacje są mało stabilne, a większe jedynie spowalniają proces.
% Stosunek nie wydaje się być liniowy, ale mamy za mało danych żeby to potwierdzić
\\

Dla tempa wymiany równego 45 przeprowadzone zostały symulacje \ref{tab:3-45-180}
do \ref{tab:3-45-315}.

\inputgraph{3-45-180.transient.tex}
\inputgraph{3-45-225.transient.tex}
\inputgraph{3-45-270.transient.tex}
\inputgraph{3-45-315.transient.tex}

Dla tempa wymiany równego 30 przeprowadzone zostały symulacje \ref{tab:3-30-60}
do \ref{tab:3-30-360} z populacjami od 60 do 360.

\inputgraph{3-30-60.transient.tex}
\inputgraph{3-30-90.transient.tex}
\inputgraph{3-30-120.transient.tex}
\inputgraph{3-30-150.transient.tex}
\inputgraph{3-30-180.transient.tex}
\inputgraph{3-30-210.transient.tex}
\inputgraph{3-30-240.transient.tex}
\inputgraph{3-30-300.transient.tex}
\inputgraph{3-30-360.transient.tex}

Dla tempa wymiany równego 15 przeprowadzone zostały symulacje \ref{tab:3-15-60}
do \ref{tab:3-15-105}.

\inputgraph{3-15-60.transient.tex}
\inputgraph{3-15-75.transient.tex}
\inputgraph{3-15-90.transient.tex}
\inputgraph{3-15-105.transient.tex}

Dla tempa wymiany równego 10 przeprowadzone zostały symulacje \ref{tab:3-10-40}
do \ref{tab:3-10-70}.

\inputgraph{3-10-40.transient.tex}
\inputgraph{3-10-50.transient.tex}
\inputgraph{3-10-60.transient.tex}
\inputgraph{3-10-70.transient.tex}

\subsection{metody selekcji}
Metoda wybierania które osobniki rozmnożą się i które przeżyją.
w symulacji \ref{tab:4-ranked} użyta jest metoda rankingowa, w symulacjach
\ref{tab:4-roulette1} do \ref{tab:4-roulette20} każdemu osobnikowi przypisywany
jest wycinek koła odpowiadający jego przystosowaniu, jednak ze względu na dość
małe różnice pomiędzy poszczególnymi osobnikami nie przynosiło to zbyt dobrych
wyników w symulacji \ref{tab:4-roulette1}, dlatego w symulacjach
\ref{tab:4-roulette5} do \ref{tab:4-roulette20} przystosowanie jednostki zostało
podniesione do 5, 10, 15 i 20 potęgi. w symulacji \ref{tab:4-top} funkcja wyboru
zawsze zwraca najbardziej przystosowane jednostki.

\inputgraph{4-ranked.transient.tex}
\inputgraph{4-roulette1.transient.tex}
\inputgraph{4-roulette5.transient.tex}
\inputgraph{4-roulette10.transient.tex}
\inputgraph{4-roulette15.transient.tex}
\inputgraph{4-roulette20.transient.tex}
\inputgraph{4-roulette25.transient.tex}
\inputgraph{4-roulette30.transient.tex}
\inputgraph{4-top.transient.tex}

\subsection{metody krzyżowania}
w symulacjach \ref{tab:5-5} do \ref{tab:5-6} porównaliśmy wpływ metody
krzyżowania na wyniki. Nieprzewidywalność krzyżowania wydaje się pozytywnie
wpływać na wyniki dla przykładowego zestawu danych.
\begin{itemize}
	\item \cljt{stripe-cross} \ref{tab:5-5} --- wybiera zawsze ten sam zestaw cech z jednego osobnika
	\item \cljt{one-point} \ref{tab:5-1} -- jest spora korelacja wśród poszczególnych cech
	\item \cljt{two-point} \ref{tab:5-2} -- korelacja jest mniejsza
	\item \cljt{random-cross} \ref{tab:5-6} -- dla każdej cechy rodzic źródłowy jest losowany
	      oddzielnie; brak korelacji
\end{itemize}

\inputgraph{5-5.transient.tex}
\inputgraph{5-1.transient.tex}
\inputgraph{5-2.transient.tex}
\inputgraph{5-6.transient.tex}
\subsection{funkcja przystosowania}
symulacje \ref{tab:6-1} do \ref{tab:6-11} przedstawiają różnice w wynikach
spowodowane różnymi funkcjami przystosowania. \\
Wszystkie zaimplementowane funkcje przyznają osobnikom które nie przekroczyły maksymalnej wagi
punkty równe ich wartości, różnią się jedynie zachowaniem dla osobników o zbyt
dużej wadze.
Symulacja \ref{tab:6-1} ocenia je na 0, symulacje \ref{tab:6-4} do \ref{tab:6-6} dzielą ich wynik przez trzecią potęgę stosunku
nadwagi do wagi maksymalnej i mnożą przez odpowiednio 0.95, 1 i 1.05.
Wyniki symulacji \ref{tab:6-6} są dziwne ponieważ dawany w niej był priorytet
osobnikom przekraczającym maksymalną wagę w związku z tym znaczna część
populacji była niepoprawna. Wśród nich najlepsze wyniki
zwróciła symulacja \ref{tab:6-5} która nie dawała preferencji osobnikom po
żadnej stronie. w symulacjach \ref{tab:6-7}, \ref{tab:6-11} podobnie jak
\ref{tab:6-5} osobniki przekraczające maksymalną masę nie mają skalowanej
punktacji, jednak został zredukowana szybkość spadku punktów z powodu przekroczenia
granicy, w \ref{tab:6-11} powyżej maksymalnej wartości punktacja zależy jedynie
od stosunku wartości do wagi. Wysoką medianę można wyjaśnić przez niepoprawność
znacznej części populacji.
\inputgraph{6-1.transient.tex}
\inputgraph{6-4.transient.tex}
\inputgraph{6-5.transient.tex}
\inputgraph{6-6.transient.tex}
\inputgraph{6-7.transient.tex}
\inputgraph{6-8.transient.tex}
\inputgraph{6-9.transient.tex}
\inputgraph{6-10.transient.tex}
\inputgraph{6-11.transient.tex}
\end{document}
