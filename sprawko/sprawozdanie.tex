\documentclass{article}
\usepackage[a4paper]{geometry}

\usepackage{amsmath} \usepackage{amssymb} \usepackage{amsfonts}
\usepackage{pifont}
\usepackage{enumitem}
\usepackage[T1]{fontenc}
\usepackage{lmodern}
\usepackage{rotating}
\usepackage{siunitx}
\usepackage[polish]{babel}
\usepackage[utf8]{inputenc}
\usepackage{multirow}
\usepackage[yyyymmdd]{datetime} \renewcommand{\dateseparator}{-} % ISO-8601
\usepackage{tabularx}
\usepackage{graphicx}
\usepackage{float}
\usepackage{minted}
\begin{document}
\begin{minipage}{0.35\linewidth}
	\begin{tabular}{lr}
		Antoni Jończyk & 236551 \\
		Tomasz Roske   & 236639
	\end{tabular} \hfill
\end{minipage}
\hfill
\begin{minipage}{0.35\linewidth}
	\hfill Rok akademicki 2022/23 \par
	\hfill czwartek, 13:00
\end{minipage}
\bigskip \bigskip \bigskip \bigskip \bigskip
\begin{center}
	\textbf{Metaheurystyki i ich zastosowania, zadanie 2}\\
	\bigskip
	\large nie wiem co zrobić mamy bo mi wikamp padł
\end{center}
\bigskip \bigskip
\section{Działanie programu}
\inputminted{clojure}{snippets/alg.clj_advance}
\subsection{krzyżowanie}
Funkcje krzyżujące zwracają listę funkcji dwuargumentowych, które są aplikowane
do kolejnych wyborów krzyżowanych osobników, lista wyników tych wywołań tworzy
genom nowego osobnika.

\begin{table}[H]
	\caption{Wzory do niepewności granicznej pomiarów}

	\begin{minipage}{0.55\textwidth}
		\resizebox{\textwidth}{!}{
			\begin{tabular}{|c|c|c|c|c|c|}
				\hline
				{Pokolenie} & max      & Q3       & Q2       & Q1       & min      \\
				\hline
				50          & 100.00\% & 100.00\% & 100.00\% & 100.00\% & 100.00\% \\
				100         & 5        & 4        & 3        & 2        & 1        \\
				150         & 5        & 4        & 3        & 2        & 1        \\
				200         & 5        & 4        & 3        & 2        & 1        \\
				300         & 5        & 4        & 3        & 2        & 1        \\
				400         & 5        & 4        & 3        & 2        & 1        \\
				500         & 5        & 4        & 3        & 2        & 1        \\
				600         & 5        & 4        & 3        & 2        & 1        \\
				700         & 5        & 4        & 3        & 2        & 1        \\
				800         & 5        & 4        & 3        & 2        & 1        \\
				\hline
			\end{tabular}
		}
	\end{minipage}
	\begin{minipage}{0.45\textwidth}
		\includegraphics[width=\textwidth]{uwu.pdf}
		\label{tab:niepewności}
	\end{minipage}
\end{table}



\end{document}
