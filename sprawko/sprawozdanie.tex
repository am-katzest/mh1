\documentclass{article}
\usepackage[a4paper,margin=0.8in]{geometry}

\usepackage{amsmath} \usepackage{amssymb} \usepackage{amsfonts}
\usepackage{pifont}
\usepackage{enumitem}
\usepackage[T1]{fontenc}
\usepackage{lmodern}
\usepackage{rotating}
\usepackage{siunitx}
\usepackage[polish]{babel}
\usepackage[utf8]{inputenc}
\usepackage{multirow}
\usepackage[yyyymmdd]{datetime} \renewcommand{\dateseparator}{-} % ISO-8601
\usepackage{tabularx}
\usepackage{graphicx}
\usepackage{float}
\usepackage{minted}

\begin{document}
\begin{minipage}{0.35\linewidth}
	\begin{tabular}{lr}
		Antoni Jończyk & 236551 \\
		Tomasz Roske   & 236639
	\end{tabular} \hfill
\end{minipage}
\hfill
\begin{minipage}{0.35\linewidth}
	\hfill Rok akademicki 2022/23 \par
	\hfill czwartek, 13:00
\end{minipage}
\bigskip \bigskip \bigskip \bigskip \bigskip
\begin{center}
	\textbf{Metaheurystyki i ich zastosowania, zadanie 2}\\
	\bigskip
	\large implementacja algorytmu genetycznego
\end{center}
\bigskip \bigskip
\section{Działanie programu}
%\inputminted{clojure}{snippets/alg.clj_advance}
\subsection{krzyżowanie}
Funkcje krzyżujące zwracają listę funkcji dwuargumentowych, które są aplikowane
do kolejnych wyborów krzyżowanych osobników, lista wyników tych wywołań tworzy
genom nowego osobnika.
\newpage



\subsection{wyniki}
\subsubsection{łączenie}
tutaj mamy z łączeniem \\
\input{1-mit.transient.tex}
a tu bez \\
\input{1-mif.transient.tex}
\subsubsection{prawdopodobieństwo mutacji}
\input{2-m0.transient.tex}
10 to wartość domyślna dla innych grafów
\input{2-m10.transient.tex}
\input{2-m25.transient.tex}
\input{2-m50.transient.tex}
\input{2-m75.transient.tex}
\input{2-m100.transient.tex}
\subsubsection{wielkość populacji}
\input{3-25.transient.tex}
\input{3-50.transient.tex}
\input{3-100.transient.tex}
\input{3-200.transient.tex}
\input{3-400.transient.tex}
\input{3-800.transient.tex}
\subsubsection{metody selekcjji}
\input{4-ranked.transient.tex}
\input{4-roulette1.transient.tex}
\input{4-roulette5.transient.tex}
\input{4-roulette10.transient.tex}
\input{4-roulette15.transient.tex}
\input{4-roulette20.transient.tex}
\input{4-top.transient.tex}
\subsubsection{metody krzyżowania}
\input{5-1.transient.tex}
\input{5-2.transient.tex}
\input{5-5.transient.tex}
\input{5-6.transient.tex}
\end{document}
