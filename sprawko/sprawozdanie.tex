\documentclass{article}
\usepackage[a4paper,margin=0.8in]{geometry}

\usepackage{amsmath} \usepackage{amssymb} \usepackage{amsfonts}
\usepackage{pifont}
\usepackage{enumitem}
\usepackage[T1]{fontenc}
\usepackage{lmodern}
\usepackage{rotating}
\usepackage{siunitx}
\usepackage[polish]{babel}
\usepackage[utf8]{inputenc}
\usepackage{multirow}
\usepackage[yyyymmdd]{datetime} \renewcommand{\dateseparator}{-} % ISO-8601
\usepackage{tabularx}
\usepackage{graphicx}
\usepackage{float}
\usepackage{minted}
\newcommand{\cljt}[1]{\mintinline{clojure}{#1}}
\begin{document}
\begin{minipage}{0.35\linewidth}
	\begin{tabular}{lr}
		Antoni Jończyk & 236551 \\
		Tomasz Roske   & 236639
	\end{tabular} \hfill
\end{minipage}
\hfill
\begin{minipage}{0.35\linewidth}
	\hfill Rok akademicki 2022/23 \par
	\hfill czwartek, 13:00
\end{minipage}
\bigskip \bigskip \bigskip \bigskip \bigskip
\begin{center}
	\textbf{Metaheurystyki i ich zastosowania, zadanie 2}\\
	\bigskip
	\large implementacja algorytmu genetycznego
\end{center}
\bigskip \bigskip
\section{Działanie programu}

\subsection{struktury danych}
\begin{itemize}
	\item osobnik \cljt{specimen} zawiera listę cech, wagę, wartość i flagę
	      wyznaczającą poprawność
	\item stan świata \cljt{state} to zbiór osobników
\end{itemize}
\subsection{symulacja}
wynikiem symulacji jest lista stanów świata z poszczególnych pokoleń
\inputminted{clojure}{snippets/alg.clj_simulate}
\inputminted{clojure}{snippets/alg.clj_advance}
początkowe pokolenie zawiera osobniki z cechami z równą szansą na bycie 0 i 1
\inputminted{clojure}{snippets/alg.clj_orphan}
%\inputminted{clojure}{snippets/alg.clj_specimen} % zbyt niski poziom

\subsection{wybór}
do zadecydowania które osobniki przetrwają lub rozmnorzą się wykorzystywana jest
jedna z trzech funkcji; \cljt{ranked} przypisująca osobnikom szansę równą ich
miejscu w rankingu, \cljt{roulette}, przypisująca osobnikom szansę równą ich
przystosowaniu  albo \cljt{top} która zawsze wybiera \cljt{n} najbardziej
przystosowanych osobników
\inputminted{clojure}{snippets/alg.clj_top}
\subsection{funkcja przystosowania}
Postanowiliśmy dać szansę osobnikom, które pomimo nieznacznego przekroczenia dopuszczalnej
wagi mają duży stosunek wartości do wagi.
\inputminted{clojure}{snippets/alg.clj_scoring}
\subsection{krzyżowanie}
Funkcje krzyżujące zwracają listę funkcji dwuargumentowych, które są aplikowane
do kolejnych wyborów krzyżowanych osobników, lista wyników tych wywołań tworzy
genom nowego osobnika.
\inputminted{clojure}{snippets/alg.clj_krzyżowanie}

\newpage



\section{analiza wyników}
\subsection{co to jest i jak to czytać}
Osobniki które przekroczyłymaksymalną masę nie są uwzględniane w wynikach.
\\
Na grafach pokazany jest rozkład (pięć kwantyli) maksimów i median każdej
symulacji, oś X reprezentuje pokolenie, a Y wartość.
\\
W tabelach przedstawiono rozkład przystosowania najlepszej utworzonej do danego pokolenia
(nawet jeżeli umarła) jednostki dla poszczególnych symulacji.
\\
W kolumnach min, Q2 i max są odpowiednie wartości (jako procent maksimum globalnego) danych kwantyli.
\\
W kolumnie \%global jest procent symulacji które znalazły maksimum globalne, a
w \%$\le$1\% jest procent symulacji które znalazły kombinację cech, z wartością w
granicach 1\% od niej.

\subsection{łączenie}
ten parametr decyduje, czy symulacja pozwala na jednoczesne istnienie wielu
identycznych jednostek.\\
w symulacjach \ref{tab:1-mit} każda jednostka musi być wyjątkowa, a w
\ref{tab:1-mif} nie. połączenie się rozkładu mediany i maksimum na wykresie
\ref{tab:1-mif} wskazuje na dominację populacji przez identyczne jednostki
z lokalnego maksimum. Ponieważ w \ref{tab:1-mit} wyniki są lepsze, ten wybór
jest zastosowany przy innych próbach.

\input{1-mit.transient.tex}
\input{1-mif.transient.tex}
\subsection{prawdopodobieństwo mutacji}
\input{2-m0.transient.tex}
\input{2-m10.transient.tex}
\input{2-m25.transient.tex}
\input{2-m50.transient.tex}
\input{2-m75.transient.tex}
\input{2-m100.transient.tex}
\subsection{wielkość populacji}

tempo wymiany=30, populacja od 60 do 360
widzimy, że najszybszy wzrost zachodzi dla
\input{3-30-60.transient.tex}
\input{3-30-90.transient.tex}
\input{3-30-120.transient.tex}
\input{3-30-150.transient.tex}
\input{3-30-180.transient.tex}
\input{3-30-210.transient.tex}
\input{3-30-240.transient.tex}
\input{3-30-300.transient.tex}
\input{3-30-360.transient.tex}

tempo wymiany=15
\input{3-15-60.transient.tex}
\input{3-15-75.transient.tex}
\input{3-15-90.transient.tex}
\input{3-15-105.transient.tex}

tempo wymiany=45
\input{3-45-180.transient.tex}
\input{3-45-225.transient.tex}
\input{3-45-270.transient.tex}
\input{3-45-315.transient.tex}

tempo wymiany=10
\input{3-10-40.transient.tex}
\input{3-10-50.transient.tex}
\input{3-10-60.transient.tex}
\input{3-10-70.transient.tex}

\subsection{metody selekcjji}
\input{4-ranked.transient.tex}
\input{4-roulette1.transient.tex}
\input{4-roulette5.transient.tex}
\input{4-roulette10.transient.tex}
\input{4-roulette15.transient.tex}
\input{4-roulette20.transient.tex}
\input{4-top.transient.tex}
\subsection{metody krzyżowania}
\input{5-1.transient.tex}
\input{5-2.transient.tex}
\input{5-5.transient.tex}
\input{5-6.transient.tex}
\subsection{funkcja przystosowania}
\input{6-1.transient.tex}
\input{6-4.transient.tex}
\input{6-5.transient.tex}
\input{6-6.transient.tex}
\input{6-7.transient.tex}
\input{6-8.transient.tex}
\subsection{szybkośćowy}
\input{7-1.transient.tex}
\input{7-2.transient.tex}
\end{document}
